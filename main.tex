\documentclass{article}
\usepackage{times}
\usepackage[margin=2.5cm]{geometry}
%\usepackage[a4paper]{geometry}
\usepackage{graphicx}
\usepackage[dutch]{babel}
\begin{document}
\centerline{\Huge Vooruit met de NLUUG!}
$ $\\
\centerline{\Large Een voorstel van Koen de Jonge en Ronny Lam}
$ $\\
\centerline{Versie 0.2 d.d. 30-04-2019} \\

% Versiebeheer
%0.1 29-04-2019 Omzetting vanuit mail feedbackronde
%0.2 30-04-2019 Signaalwoorden van capital naar vet. Diverse spelfouten aangepast.

\section{TL;DR}
Wij willen vooruit met de NLUUG en beschrijven hier hoe we dat 
willen. Onderdeel daarvan is dat we voorstellen om het bestuur van de vereniging mede vorm te geven.

\section{Inleiding}
De vereniging Nederlandse Lokale Unix-Systems Gebruikers Groep (NLUUG) heeft nog steeds bestaansrecht. Onze voor- en najaarsconferenties worden nog steeds goed  bezocht en het aantal leden neemt al weer jaren toe. Er is een vaste groep mensen van rond de 500 mensen die regelmatig naar de conferentie komen. Ook de FTP-server van de NLUUG wordt meer en meer gebruikt, al is het misschien niet voor alle gebruikers duidelijk dat er een link is met onze vereniging.\\

In 2016 is er besloten tot een kern-bestuur bestaande uit een voorzitter, secretaris en penningmeester. Het idee hiervan was dat het bestuur zich zou beperken tot besturen en inhoudelijke activiteiten bij de commissies zou laten. Wij denken dat het nu nodig is om de organisatorische processen en kaders van de vereniging verder uit te werken zodat er voldoende ruimte ontstaat om commissies grotendeels zelf aan de slag te kunnen laten gaan. Daarvoor is naar ons inzicht de afgelopen jaren onvoldoende ruimte geweest.\\

Op de volgende vlakken denken we dat het bestuur nog veel werk te verzetten heeft:
\begin{itemize}
    \item Financi\"ele organisatie (op orde krijgen van de boekhouding, innen van de contributie en plannen van de uitgaven)
    \item Administratieve organisatie (op orde krijgen van de ledenadministratie, afronden nieuwe lidmaatschapsstructuur)
    \item  Communicatieve organisatie (op orde krijgen van communicatiekanalen met leden en commissies, mailinglijsten, digitale ondersteuning voor het bijhouden van vergaderingen e.d.)
    \item Promotionele organisatie (op orde krijgen van de website, communicatie met de media, samenwerking met andere verenigingen)
    \item  Relationele organisatie (communicatie met donateurs, sponsoren, kennisinstituten, gerelateerde verenigingen, politiek)
\end{itemize}

\section{Golden circle}
Om dit goed uit te werken is het nodig dat we als leden weten waar de vereniging voor staat. Je zou dat kunnen doen door een missie, visie en strategie te bepalen, maar wij werken graag met de indeling van Simon Sinek. Voor het onderstaande hebben we voor een groot deel gebruik gemaakt van bestaande stukken en inzichten.

\subsection{WHY (Waarom bestaat de NLUUG?)}
De NLUUG heeft haar bestaansrecht in de volgende punten:
\begin{enumerate}
    \item De NLUUG is de \textbf{autoriteit} in Nederland op het gebied van Informatietechnologie, Open Source, Open Systemen en Open Standaarden.
    \item De NLUUG vormt een onge\"evenaarde \textbf{kennisbron} door op een open manier verbinding te realiseren met een professionele insteek.
    \item De NLUUG staat voor het open en vrij \textbf{delen} van kennis en informatie over systemen en technologie.
    \item De NLUUG is \textbf{onafhankelijk} ten opzichte van bedrijven, opleidingsinstituten, (open source) softwareprojecten, politiek en pers en vormt daardoor een onmisbare spil in de samenleving.
\end{enumerate}

\subsection{HOW (Hoe vervult de NLUUG deze rol?)}
De NLUUG biedt individuen en organisaties mogelijkheden om aan kennisdeling op het vlak van Informatietechnologie, Open Source, Open Systemen en Open Standaarden te doen door:
\begin{enumerate}
    \item Het organiseren van conferenties, presentaties en andere (virtuele) bijeenkomsten.
    \item Het bieden van platforms voor het delen van software, informatie en kennis (vaardigheden).
    \item Het verzamelen en delen van verwijzingen (Nederlands en Internationaal) nieuws.
    \item Het bijhouden en publiceren van een agenda met evenementen en gebeurtenissen.
    \item Het faciliteren van communicatie en delen van inhoudelijke kennis en visie aan de samenleving.
    \item Het faciliteren en organiseren van samenwerking met individuen, opleidingsinstituten en bedrijven op het punt van bijvoorbeeld ontwikkeling van open source software en open standaarden. Met name op gebieden waar de behoefte van de samenleving wel aanwezig is maar dit niet van de grond komt door gebrek aan directe-waardeontwikkeling voor de genoemde partijen.
\end{enumerate}

De NLUUG streeft naar een ledenbestand met zo veel mogelijk inclusiviteit met dekking door de hele samenleving. \\

De NLUUG wil haar ledenbestand de komende 10 jaar met 10\% per jaar laten groeien zodat er meer en grotere impact bereikt kan worden in de samenleving. De vereniging wil meer `naar buiten gericht' groeien waardoor meer leden geworven worden, maar ook meer `naar binnen gericht' (inhoudelijk) groeien waardoor de leden meer waarde ervaren aan de vereniging. \\

De NLUUG wil aan haar leden de mogelijkheid bieden zich in lijn met de visie en missie te ontplooien en geeft hieraan invulling door individuele leden financieel en organisatorisch te ondersteunen. \\

In de visie en missie van de NLUUG is zo min mogelijk onderscheid gemaakt in deelgebieden zoals security, software ontwikkeling, monitoring, auditing, systeembeheer, netwerken, storage, IT-Infrastructuur in het algemeen, community-projecten, privacy, etc. Toch vallen al deze deelgebieden onder de algemene termen die gebruikt zijn. \\

\subsection{WHAT (Wat gaat de NLUUG doen om dit waar te maken?)}
We willen de WHY en HOW in het komende jaar (2019) verder uitwerken en komen tot een duidelijk uitvoerbare `wat gaan we doen' om de basis te leggen voor een reeks aan actieve besturen voor de vereniging. Door een duidelijke en ambitieuze taak voor deze besturen te formuleren willen we het bestaansrecht van de vereniging voor de komende 10 jaar garanderen en de potentie van de vereniging maximaliseren. \\

Wij stellen voor om het bestuur tijdens de komende ALV van 23 mei weer te laten groeien naar 5 mensen waardoor er meer `mens-tijd' is om de uitwerking van de hier geformuleerde stappen te kunnen laten plaatsvinden. We willen dat dit bestuur zich meer ori\"enteert en helder profileert samen met de ledenvergadering. \\

We willen doeltreffend en doortastend handelen om de financi\"ele organisatie van de vereniging op orde te krijgen. Door met het bestuur samen met actieve leden een actieplan op te stellen willen we voor de volgende ledenvergadering (in het najaar) orde op zaken stellen. Samen met de actieve leden, wat inhoudt in dat wie zich aanmeld in volledige openheid mag meedenken, meekijken en meewerken. Zo willen we met behoud van leden, donateurs of sponsoren `in the process' leden heractiveren, opnieuw interesseren en weer deel(genoot) maken van de vereniging. Hiervoor willen we zowel de bestaande middelen aanwenden als nieuwe middelen aantrekken. Dit door vroegere donateurs en sponsoren te heractiveren en nieuwe sponsoren en donateurs actief te werven. Een belangrijk aandachtspunt hierbij op korte termijn is het aantrekken en behouden van jongere nieuwe leden. Dat kan bijvoorbeeld door ons te richten op mensen die aan het begin van hun carri\`ere staan. \\

We willen dat dezelfde groep mensen (bestuur en actieve leden) een PR/Marketing plan formuleert en een frisse nieuwe huisstijl laat ontwikkelen en op basis daarvan een nieuwe website ontwikkelt. Mogelijk deels door inhuur van externe, aan de vereniging gelieerde bedrijven of individuen) met behoud van alle bestaande inhoud. Waar mogelijk wordt informatie en kennis die nu besloten is publiek gemaakt. \\

We denken dat het werven van leden en donateurs een belangrijke aparte taak is waar het verstandig is om de focus door een of meerdere personen aan toe te kennen. We denken ook dat het werven van sponsors zo een aparte taak is. Omdat beide taken zeer nauw verworven zijn met het bestuur van de vereniging denken we dat voor beide rollen een aparte bestuursfunctionaris gewenst is. \\

We willen, met het oog op de privacy en recent daarop ontstane nieuwe regels en grenzen, leden en potenti\"ele leden de (opnieuw) mogelijkheid bieden om, al dan niet anoniem, met elkaar in contact te treden en kennis en informatie en vaardigheden te delen zonder privacy en andere rechten te verliezen. Hiertoe zal actief toenadering gezocht worden tot initiatieven zoals IRMA, ToR en organisaties zoals BoF en ISOC, maar ook Surf/Sara, NLnet en hackerspaces. Ook en specifiek voor deze `communicatie-facilitering' willen we onderzoeken hoe we financi\"ele en technische middelen hiervoor beschikbaar kunnen maken. We denken hierbij bijvoorbeeld om sponsors, subsidies en fondsen te koppelen aan (nieuwe) NLUUG commissies. \\

We willen onderzoeken wat de mogelijkheden zijn om met ingang van 2020 of 2021 een of meerdere financieel ondersteunde functies binnen de NLUUG beschikbaar te stellen zodat het mogelijk wordt om meer kwaliteit en prioriteit te kunnen geven aan de werkzaamheden voor de vereniging. \\

We willen onderzoeken wat de mogelijkheden zijn om meer fondsen te werven en grotere (risicovollere) activiteiten te ondernemen zonder het bestaan van de vereniging te bedreigen. Hierbij valt te denken aan het (laten) ontwikkelen van verenigingsondersteunende software voor communicatie-doeleinden of het organiseren van zeer zichtbare `marketing-campagnes' die de doelen, visie en missie van de vereniging ondersteunen. \\

\section{Ter afsluiting}
We realiseren ons dat dit allemaal zeer ambitieus is en dat we mogelijk alleen maar een begin kunnen maken met het op orde stellen van zaken en het schetsen van plannen. Toch denken we dat er veel meer mogelijk is voor de NLUUG door de ambitie uit te spreken, beter te communiceren, mensen meer de kans te geven om actief aan de slag te gaan en fouten te maken en vooral door er samen, met en door alle leden voor te gaan. Vooruit met de NLUUG!
\end{document}
